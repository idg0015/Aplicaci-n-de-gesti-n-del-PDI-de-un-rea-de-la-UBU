\capitulo{7}{Conclusiones y Líneas de trabajo futuras}

Para finalizar la memoria de este proyecto, se va a hacer una valoración final del trabajo realizado y los conocimientos adquiridos durante el proceso.

También se van a indicar nuevas funcionalidades que podrían ser interesantes para la aplicación desarrollada y que harían esta herramienta un \textit{software} mucho más completo.

\section{Conclusiones}
Este trabajo de fin de grado pone punto final a cuatro años de formación y aprendizaje, no solo el desarrollo de \textit{software}, si no de haber adquirido una visión crítica sobre cualquier fase del ciclo de vida de un proyecto, enfocado sobre todo al ámbito de la informática.

La realización de este trabajo ha permitido plasmar la capacidad adquirida de adaptación a cualquier tipo de lenguaje o tecnología, y la aptitud de seguir el proceso de desarrollo de un proyecto desde su diseño hasta su despliegue.

La redacción de la documentación ha sido una de las partes que más recursos temporales ha consumido, pero que es necesario para plasmar el trabajo realizado, los conocimientos adquiridos y marcar unas líneas de trabajo y manuales para que futuros alumnos puedan continuar con el desarrollo de este proyecto.

"Considero importante resaltar el trabajo de investigación y aprendizaje llevado a cabo durante el proyecto, incluso antes de su asignación oficial.
Las investigaciones realizadas no han sido sobre artículos u otros trabajos anteriores, sino que se ha realizado un proceso de investigación de herramientas y técnicas que podrían resultar útiles, así como el aprendizaje de su correcto uso.

Durante la vida del proyecto también se han tenido que aprender y aplicar buenas técnicas de programación para intentar hacer el código lo más accesible, mantenible y escalable posible.

Además de todo lo anterior, el desarrollo web es una parte de la informática que se aborda en menor medida a lo largo de la carrera. Por lo tanto, llevar a cabo un proyecto que se centre principalmente en el desarrollo web implica la necesidad de aprender sobre este tema en poco tiempo.


\section{Líneas de trabajo futuras}
A continuación, se expondrán una serie de mejoras que podrían llevar a la aplicación a un nivel aún más completo.

\subsection{Cambios en las áreas}
Posiblemente será necesaria una modificación del diseño de áreas debido al cambio que se pretende introducir con el <<Proyecto de Real Decreto por el que se establecen los ámbitos de conocimiento a efectos de la adscripción de los puestos de trabajo del profesorado universitario>>~\cite{gob:areas}. 

Entre las medidas incluidas en el proyecto, se encuentra una que pretende cambiar el nombre que reciben las áreas en la descripción de las plazas de profesorado.

\subsection{Creación avanzada de grupos}
En la actualidad, los grupos se generan automáticamente con un nombre basado en su tipo (teórico o práctico) y modalidad (online, presencial o en inglés). Este enfoque asegura el uso correcto de los códigos de grupo y mantiene una proporción adecuada entre los grupos teóricos y prácticos.

No obstante, sería interesante considerar la adición de una funcionalidad que permita elegir la forma de crear un nuevo grupo. De esta manera, se podría mantener el sistema actual, al mismo tiempo que se incorporaría una nueva opción que permita agregar un grupo manualmente mediante la introducción del código correspondiente, sin afectar al resto de los códigos de grupo.

\subsection{Añadir gráficas}
La aplicación desarrollada permite obtener diversos datos, como las horas que un docente imparte en un curso o titulación, la cantidad de plazas por cada tipo de contrato o el número de contrataciones en un rango de fechas, entre otros.

Toda esta información podría ser representada en diferentes gráficas. Además, sería posible hacer que estas gráficas sean interactivas, permitiendo al usuario agregar o eliminar fuentes de datos para obtener la representación gráfica deseada.

Considero que esta mejora sería muy interesante, ya que proporcionaría información de manera más rápida y visual que en la configuración actual.

\subsection{Integración con Moodle}
Otro aspecto que podría ser muy interesante desarrollar es buscar la manera de integrar esta aplicación con Moodle, la plataforma utilizada por la universidad.

Moodle ofrece una API que permite realizar diversas acciones. 
Sería beneficioso explorar en mayor profundidad esta API para determinar la posibilidad de recuperar información relevante del Moodle, como datos de docentes, titulaciones, asignaturas, entre otros. 
Asimismo, se podría agregar nueva información a la aplicación que también pudiera obtenerse desde esta plataforma.

Por otro lado, se podría estudiar la funcionalidad inversa, es decir, la capacidad de enviar datos desde la aplicación web desarrollada hacia Moodle.
De esta forma se podría, por ejemplo, realizar la planificación de un curso académico desde esta aplicación y que se replique de forma automática en Moodle.

Cabe destacar que esta nueva funcionalidad requeriría un estudio e integración exhaustivos, ya que es probable que los datos no se envíen en el mismo formato y se requieran diferentes soluciones.
No obstante, creo que podría ser una característica muy cómoda para el uso real de la aplicación.

\subsection{Generación de horarios}
Una interesante adición sería la capacidad de generar horarios dentro de la aplicación. 
Gracias a la información proporcionada sobre titulaciones, asignaturas y grupos asociados, sería posible generar automáticamente un horario que aportando las horas de las clases.

Además, al especificar los docentes asignados a cada grupo desde la aplicación, también podrían aparecer en el horario correspondiente. 
Esto permitiría generar un horario completo que, incluso, podría convertirse a formato PDF para facilitar su visualización y distribución.


