\capitulo{1}{Introducción}

La gestión del personal docente e investigador es una tarea crítica y compleja para cualquier institución de educación superior, ya que implica el manejo de información confidencial y la coordinación de múltiples actividades administrativas. 
Por lo tanto, contar con una herramienta informática eficiente y segura para gestionar esta tarea es fundamental.

El presente trabajo de fin de grado tiene como objetivo la creación de una aplicación web para la gestión del personal docente e investigador (PDI) de la Universidad de Burgos utilizando el lenguaje de programación Python y el \textit{framework} Flask.

En este contexto, se ha desarrollado una aplicación web que permite la gestión de información de los docentes e investigadores, así como el acceso a diferentes funcionalidades de forma sencilla y eficiente. 

Este trabajo describe el proceso de desarrollo de la aplicación, desde la definición de los requisitos y la arquitectura de la solución, hasta la implementación, despliegue y evaluación de la aplicación. 

Una de las principales características, a la que se ha prestado una especial atención, es el diseño de la aplicación.
Se ha procurado desarrollar la aplicación web siguiendo buenas prácticas que aseguren una base útil para poder ser fácilmente extensible y mantineble.
Además, de cara a la interfaz, se ha procurado mantener un diseño agradable y minimalista que facilite el trabajo y que se adapte a cualquier resolución de pantalla.

\section{Estructura}
Podemos diferenciar en la documentación del proyecto la memoria y los anexos.

En la memoria se encuentra la información principal de proyecto realizado mientras que en los anexos se puede ver toda la información relacionada con el desarrollo y despliegue de la aplicación realizada.

\subsection{Memoria}
La memoria está compuesta por las siguientes secciones:
\begin{itemize}
\item \textbf{Introducción:} Explicación breve sobre el proyecto realizado y la estructura.
\item \textbf{Objetivos del proyecto:} En esta sección se exponen las metas que se quieren lograr con la realización del proyecto.
\item \textbf{Conceptos teóricos:} Algunas explicaciones necesarias para comprender la totalidad del proyecto desarrollado.
\item \textbf{Técnicas y herramientas:} Descripción de las principales técnicas y herramientas utilizadas durante el diseño y desarrollo del proyecto.
\item \textbf{Aspectos relevantes del desarrollo del proyecto:} Exposición de aquellos elementos más relevantes del proyecto.
\item \textbf{Trabajos relacionados:} Enumeración y descripción breve de algunos trabajos similares a este.
\item \textbf{Conclusiones y líneas de trabajo futuras:} Resumen del trabajo realizado junto a las experiencias obtenidas gracias a él y algunas mejorar que podrían ser incluidas en futuras versiones.
\end{itemize}

\subsection{Anexos}
En los anexos encontramos las diferentes secciones que profundizan acerca del desarrollo de la aplicación:
\begin{itemize}
\item \textbf{Plan de Proyecto Software:} Estudio sobre la viabilidad económica y legal de proyecto. Además, se realiza una planificación temporal del trabajo realizado.
\item \textbf{Especificación de Requisitos:} Enumeración y estudio de los requisitos y funcionamiento del \textit{software} desarrollado.
\item \textbf{Especificación de diseño:} Análisis y desarrollo de los datos y estructuras utilizadas para el desarrollo.
\item \textbf{Documentación técnica de programación:} Recursos para facilitar el primer contacto por parte de un desarrollador con el \textit{software} del proyecto.
\item \textbf{Documentación de usuario:} Manual detallado con la información necesaria para que el usuario final maneje la aplicación.
\end{itemize}
