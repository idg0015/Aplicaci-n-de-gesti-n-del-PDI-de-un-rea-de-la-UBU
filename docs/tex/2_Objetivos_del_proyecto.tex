\capitulo{2}{Objetivos del proyecto}

Los objetivos del proyecto se centran en desarrollar una aplicación web de gestión académica y del personal docente e investigador (PDI) que satisfaga las necesidades específicas de la Universidad de Burgos. Estos objetivos se dividen en dos categorías principales: los objetivos del software a diseñar y los objetivos técnicos necesarios para llevar a cabo la implementación del proyecto.

\section{Objetivos del \textit{software}}

Los objetivos del software a diseñar e implementar son los siguientes:

\begin{enumerate}
	\item Diseñar una aplicación web intuitiva y de fácil uso, extensible y fácil de mantener para la gestión del PDI de la Universidad de Burgos.
  \item Desarrollar la aplicación web diseñada siguiendo buenas prácticas de desarrollo de \textit{software}.
  \item Centralizar la información del PDI en una base de datos segura y confiable, permitiendo un acceso eficiente y actualización de la información.
  \item Agilizar las tareas administrativas relacionadas con el PDI y el mantenimiento académico, tales como la creación de cursos académicos y la asignación de cargas docentes.
  \item Proporcionar a los usuarios diferentes funcionalidades en base a sus roles, como la consulta, creación y modificación de centros, titulaciones, asignaturas, cursos académicos, plazas, departamentos, etc.
  \item Desplegar la aplicación en la nube haciendo uso de alguna plataforma de \textit{hosting} gratuita.
\end{enumerate}

\section{Objetivos técnicos}

Los objetivos técnicos que se persiguen con la realización del proyecto son los siguientes:

\begin{enumerate}
  \item Utilizar el lenguaje de programación Python y el \textit{framework} Flask para el desarrollo de la aplicación web.
  \item Diseñar una arquitectura modular y escalable, que permita el crecimiento y la evolución futura del sistema.
  \item Integrar la aplicación con una base de datos relacional, como SQL, para el almacenamiento seguro y eficiente de la información.
  \item Aplicar buenas prácticas de desarrollo de software, como el uso de control de versiones y la documentación adecuada del código.
  \item Seguir los requisitos marcados para mejorar el software utilizado actualmente para este fin.
  \item Simular una interacción real entre cliente y programador manteniendo reuniones con las que obtener los requisitos necesarios.
  \item Documentar el diseño de la aplicación, tanto estructural como de prototipos antes de empezar la codificación.
\end{enumerate}