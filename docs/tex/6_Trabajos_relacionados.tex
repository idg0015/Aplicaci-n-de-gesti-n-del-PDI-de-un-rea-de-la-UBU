\capitulo{6}{Trabajos relacionados}

A continuación se presentarán diferentes trabajos que mantienen relación con la aplicación desarrollada en este trabajo de fin de grado.
Estos trabajos han abordado diferentes aspectos relacionados con la gestión del personal, principalmente desde el punto de vista la planificación y asignación de recursos.

\begin{itemize}
\item \textbf{SIGMA}\footnote{\url{https://www.sigmaaie.org/es}}\label{SIGMA}

SIGMA es un \textit{software} diseñado para la gestión académica y es utilizado por un gran número de universidades españolas, entre ellas la Universidad de Burgos.
Está diseñado para ayudar a las instituciones educativas a gestionar tanto sus procesos administrativos como académicos.
Esto incluye, entre otras, la gestión de estudiantes, docentes, matrículas, cursos académicos y calificaciones.

Esta plataforma es muy completa ya que permite gestionar prácticamente todo el ciclo de vida de la gestión académica. Sin embargo, cada institución tiene sus propias particularidades y ahí es donde la aplicación de este trabajo destaca por completo al estar hecha a medida para lo que se requiere.

A día de hoy SIGMA cuenta con algunas desventajas que hacen que su uso no sea del todo cómodo.
Algunas de estas son que cuenta con un diseño pensado para pantallas grandes que no se adapta bien a otro tipo de resoluciones, es decir, un diseño no \textit{responsive}, se encuentra migrada a una aplicación web de una mala manera ya que su funcionamiento es lanzando procesos en una web, no visualiza fácilmente la información en pantalla sino que se basa en listados de generación asíncrona y cuenta con problemas de integración.

\item \textbf{Classter}\footnote{\url{https://www.classter.com/}}

Es una plataforma de gestión académica que permite controlar el proceso educativo al completo.
Incluye un sistema para la gestión de estudiantes, profesorado, aulas, calificaciones, etc.

Los puntos fuertes de esta plataforma podrían ser su modularidad, que permite añadir sólo aquellas partes que se deseen, la personalización, que permite adaptar la aplicación a cada institución y la facilidad de comunicación que proporciona entre docentes y alumnos.

Entre sus principales desventajas se encuentran una gran curva de aprendizaje, es decir, que cuesta conocer y trabajar con normalidad con el sistema desde un principio y requiere de un proceso de aprendizaje previo para aprovecharlo al completo, la falta o mala integración con otras herramientas para la gestión académica y por último, el costo.

\item \textbf{Constructor}\footnote{\url{https://constructor.tech/solutions/higher-education}}

Constructor es una plataforma pensada para instituciones de educación superior.
Ofrece diferentes soluciones entre las que se encuentran la gestión del PDI, la administración de cursos, una plataforma de enseñanza \textit{online} y la gestión de servicios estudiantiles como pueden ser matrículas, biblioteca, gestión de becas, etc.


\end{itemize}
