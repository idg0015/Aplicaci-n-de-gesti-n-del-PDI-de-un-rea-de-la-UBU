\apendice{Plan de Proyecto Software}

\section{Introducción}

\section{Planificación temporal}
La planificación temporal del proyecto se ha seguido con la realización de los \textit{sprints} marcados en cada reunión.

\subsubsection{\textit{Sprint} 1: \textit{Sprint} inicial}
Fechas: 28 febrero 2023 - 7 marzo 2023
\begin{itemize}
\item\textbf{Planificación del \textit{sprint}}

En la reunión de planificación del sprint se fijaron las siguientes tareas:
\begin{enumerate}
	\item Configuración inicial del repositorio
	\item Prototipos de las vistas de la aplicación
	\item Creación diagrama entidad-relación
\end{enumerate}

\item\textbf{\textit{Burndown Report}}

\begin{figure}[h]
	\centering
	\includegraphics[width=\textwidth]{../img/Anexos/Sprints/Sprint1.png}
	\caption{\textit{Burndown Report Sprint 1}}
\end{figure}
\FloatBarrier
Como se puede apreciar en el gráfico, no todas las tareas aparecen como completadas. Esto es debido a que el diagrama entidad-relación se dejó abierto ya que faltaba información para completarlo.

\item\textbf{Revisión del \textit{sprint}}

Durante la revisión se mostró el trabajo realizado y se vieron los cambios que se debían realizar en el diagrama entidad-relación que a su vez implicaban cambios en los prototipos de las vistas de la aplicación.
Se llegó a la conclusión de que podía ser buena idea dividir el diagrama E/R haciendo vistas del mismo para que fuere más fácil resolverlo.
\end{itemize}


\subsubsection{\textit{Sprint} 2: Casos de uso y diagrama E/R de cada caso}
Fechas: 7 marzo 2023 - 14 marzo 2023
\begin{itemize}
\item\textbf{Planificación del \textit{sprint}}

En la reunión de planificación del sprint se fijaron las siguientes tareas:
\begin{enumerate}
	\item Creación de casos de uso junto a su vista del diagrama E/R
	\item Aprendizaje de \textit{Flask}
\end{enumerate}

\item\textbf{\textit{Burndown Report}}

\begin{figure}[h]
	\centering
	\includegraphics[width=\textwidth]{../img/Anexos/Sprints/Sprint2.png}
	\caption{\textit{Burndown Report Sprint 2}}
\end{figure}
\FloatBarrier
En este \textit{sprint} se completaron las tareas marcadas en el tiempo fijado durante la reunión de planificación, pero muchas cosas quedaron pendientes de cambios en futuros \textit{sprints}.

\item\textbf{Revisión del \textit{sprint}}

En la reunión de revisión se estudió de nuevo el diagrama E/R y se indicaron nuevos cambios menores en el mismo. También se propuso el comenzar a realizar el diagrama de casos de uso y continuar con el estudio de \textit{Flask}.
\end{itemize}

\subsubsection{\textit{Sprint} 3: Documentación de casos de uso e investigación y aprendizaje de \textit{Flask} y bibliotecas \textit{JavaScript}}
Fechas: 14 marzo 2023 - 21 marzo 2023
\begin{itemize}
\item\textbf{Planificación del \textit{sprint}}

En la reunión de planificación del sprint se fijaron las siguientes tareas:
\begin{enumerate}
		\item Realizar el diagrama de casos de uso
		\item Cambios en las vistas adaptándolas a los casos de uso
		\item Documentar los casos de uso con sus tablas
		\item Investigar bibliotecas de \textit{JavaScript} que pudiesen ayudar
		\item Aprender sobre \textit{Flask}
\end{enumerate}

\item\textbf{\textit{Burndown Report}}

\begin{figure}[h]
	\centering
	\includegraphics[width=\textwidth]{../img/Anexos/Sprints/Sprint3.png}
	\caption{\textit{Burndown Report Sprint 3}}
\end{figure}
\FloatBarrier
En este \textit{sprint} se completaron las tareas marcadas aunque el tiempo marcado para el aprendizaje de \textit{Flask} fue menor debido a falta de tiempo durante esta semana. Estaba previsto dedicar en total 18 horas al \textit{sprint}, pero finalmente fueron 15.

\item\textbf{Revisión del \textit{sprint}}

Durante la revisión se vio que había casos de uso que no eran necesarios y que se podían añadir como excepciones de otros. Esto produjo que el diagrama de casos de uso se debía cambiar, lo que implica un cambio en la documentación de las tablas y en los prototipos de las vistas de la aplicación.
\end{itemize}

\subsubsection{\textit{Sprint} 4: Cambios en los casos de uso y vistas, documentación y \textit{Flask}}
Fechas: 21 marzo 2023 - 28 marzo 2023
\begin{itemize}
\item\textbf{Planificación del \textit{sprint}}

En la reunión de planificación del sprint se fijaron las siguientes tareas:
\begin{enumerate}
		\item Cambios en el diagrama de casos de uso dividiendo en diagrama por niveles y crear vistas del diagrama E/R para los casos de uso.
		\item Cambios en algunos prototipos de las vistas de la aplicación.
		\item Cambios en la documentación de los casos de uso (tablas).
		\item Añadir documentación.
		\item Comenzar estructura básica de la aplicación en \textit{Flask}.
\end{enumerate}

\item\textbf{\textit{Burndown Report}}

\begin{figure}[h]
	\centering
	\includegraphics[width=\textwidth]{../img/Anexos/Sprints/Sprint4.png}
	\caption{\textit{Burndown Report Sprint 4}}
\end{figure}
\FloatBarrier
A lo largo del \textit{sprint} se realizaron la mayoría de tareas marcadas, pero en las tareas de añadir documentación y comenzar con \textit{Flask}, no se hizo tanto como estaba esperado debido a la falta de tiempo. Se planeo que se iban a poder dedicar más horas de las que al final se dedicaron y no se avanzó todo lo planeado en estas tareas. Aún así, se decidió cerrarlas, ya que algo si se había avanzado, y crear de nuevo tareas similares en próximos \textit{sprints}.
La tarea de <<Cambios en el diagrama de casos de uso dividiendo en diagrama por niveles y crear vistas del diagrama E/R para los casos de uso>> tenía una previsión de 2 horas que se realizó en el plazo estimado, la tarea <<Cambios en algunos prototipos de las vistas de la aplicación>> tenía una estimación de 3 horas aunque finalmente fueron 4, la tarea <<Cambios en la documentación de los casos de uso (tablas)>> tenía marcada 4 horas de duración que finalmente fueron 5, la tarea de <<Añadir documentación>> tenía pensada una dedicación de 3 horas que al final se quedó en 1 hora y media aproximadamente y, por último, la tarea <<Comenzar estructura básica de la aplicación en \textit{Flask}.>> tenía una previsión de 3 horas que quedó reducida a una media hora por falta de tiempo durante el \textit{sprint}.

\item\textbf{Revisión del \textit{sprint}}

En la reunión de revisión del \textit{sprint} se decidió hacer algunos cambios en los casos de uso, lo que implica un cambio en la documentación de las tablas y los prototipos de las vistas ya que el funcionamiento esperado de la aplicación cambia. También se decidió realizar algún cambio en el diagrama E/R añadiendo nuevos atributos y sacando un atributo a una nueva entidad.
\end{itemize}

\subsubsection{\textit{Sprint} 5: Documentación y comienzo de la aplicación}
Fechas: 28 marzo 2023 - 11 abril 2023
\begin{itemize}
\item\textbf{Planificación del \textit{sprint}}

En la reunión de planificación del sprint se fijaron las siguientes tareas:
\begin{enumerate}
		\item Cambios en el diagrama E/R y en algún caso de uso.
		\item Cambios en algunas tablas de los CU y en sus respectivas vistas.
		\item Evaluar plugins/frameworsk para tablas.
		\item Añadir documentación.
		\item Desarrollo de la aplicación.
\end{enumerate}

\item\textbf{\textit{Burndown Report}}

\begin{figure}[h]
	\centering
	\includegraphics[width=\textwidth]{../img/Anexos/Sprints/Sprint4.png}
	\caption{\textit{Burndown Report Sprint 5}}
\end{figure}
\FloatBarrier


\item\textbf{Revisión del \textit{sprint}}


\end{itemize}



\section{Estudio de viabilidad}

\subsection{Viabilidad económica}

\subsection{Viabilidad legal}


