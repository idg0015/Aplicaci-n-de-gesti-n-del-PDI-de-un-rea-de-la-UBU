\apendice{Especificación de diseño}

\section{Introducción}
Una vez realizado el estudio y especificación de los requisitos de la aplicación web, se debe realizar el diseño de la misma.\\
En este anexo se pretende aportar información sobre el diseño de los datos que utiliza la aplicación junto al diseño procedimental y arquitectónico del proyecto.

\section{Diseño de datos}
Gracias a la especificación de requisitos y casos de uso, se puede obtener una visión global de la aplicación que permite deducir las entidades, acompañadas de sus datos, necesarias para poder cumplir con lo requerido.\\
En primer lugar, podemos obtener la visión global de las entidades relacionadas mediante el diagrama general de Entidad-Relación de la figura~\ref{DiagramaGeneralE-R}.\\
El diagrama obtenido tiene un gran tamaño y, para mejorar la visualización y comprensión del mismo, se ha decidido dividir en vistas donde se incluyan los datos de cada una de las entidades.\\
La primera vista hace referencia al apartado de mantenimiento académico y se puede ver en la figura~\ref{er_cu1}, la segunda al mantenimiento de profesorado~\ref{er_cu2} y la última a la asignación docente~\ref{er_cu3}.

\figuraApaisadaSinMarco{}{../img/Anexos/Diagrama E-R.pdf}{Diagrama general entidad-relación}{DiagramaGeneralE-R}{}

Del diagrama entidad-relación se puede obtener el diagrama relacional de la figura~\ref{DiagramaRelacional} en el que se pueden ver la tablas que contendrá la base de datos de la aplicación web.

En este diagrama se pueden ver las tablas de la base de datos juntos los distintos campos que tendrá cada una.
Como se puede ver en la figura~\ref{DiagramaRelacional}, las tablas centro, titulación y asignatura tienen un campo llamado código. 
Este código campo podría haber sido utilizado como clave primaria, pero al ser un campo que introduce el usuario, se decidió mantener una clave primaria auto-incremental con la que se hacen las relaciones de las tablas, y además, añadir ese campo para que se puedan hacer búsquedas o filtrar por el mismo sin que su uso pueda afectar a la consistencia del sistema.

Otro aspecto relevante es que en las relaciones que dan lugar a la creación de una tabla intermedia se siguió el mismo patrón que antes. 
Aunque la teoría diga que las claves de las tablas que se relacionan pasan a ser claves primarias de la nueva tabla que se genera, se decidió tener una única clave primaria auto-incremental y tener las claves de las tablas relacionadas como claves foráneas.
De esta manera, las búsquedas en la base de datos estarán más optimizadas al buscar como clave primaria un único campo y no la composición de varios.
Además, se evita cualquier tipo de error de clave primaria al ser la propia base de datos la que asigna esta y no el código creado.

\figuraApaisadaSinMarco{}{../img/Anexos/Diagrama relacional.pdf}{Diagrama relacional}{DiagramaRelacional}{}


\section{Diseño procedimental}

\section{Diseño arquitectónico}


